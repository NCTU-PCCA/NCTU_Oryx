\section{Flow}
\subsection{Dinic}
\begin{lstlisting}
#define INF 0x3f3f3f3f
#define LINF 0x3f3f3f3f3f3f3f3fLL
struct Dinic {
	typedef long long int T;
	struct edge{
		int u, v;
		T c, f;
		edge(int _u, int _v, T _c, T _f): u(_u),v(_v),c(_c),f(_f){}
	};
	int n, s, t;
	vector<vector<int> > G;
	vector<edge> E;
	vector<int> cur, vis, d;
	Dinic(int _n):n(_n){
		G.resize(n+1);
		vis.resize(n+1); cur.resize(n+1); d.resize(n+1);
		for(int i=0; i<=n; i++)d[i] = INF;
	}
	void pb(int u, int v, T cap){
		G[u].push_back(E.size());
		E.push_back(edge(u, v, cap, 0));
		G[v].push_back(E.size());
		E.push_back(edge(v, u, 0, 0));
	}
	int bfs() {
		queue<int> q;
		for(int i=0; i<=n; i++)vis[i] = 0;
		q.push(s); d[s] = 0;
		while(!q.empty()) {
			int u = q.front(); q.pop();
			vis[u] = 1;
			for(int i=0; i<(int)G[u].size(); i++) {
				edge e = E[G[u][i]];
				if(e.c - e.f > 0 && !vis[e.v]) {
					d[e.v] = d[u] + 1;
					q.push(e.v);
				}
			}
		}
		return vis[t];
	}
	T dfs(int u, T a) {
		if(u == t || !a)return a;
		T totf = 0, f;
		for(int &i=cur[u]; i<(int)G[u].size(); i++) {
			edge &e = E[G[u][i]], &r=E[G[u][i]^1];
			if(d[e.v] != d[u]+1)continue;
			f = dfs(e.v, min(a, e.c - e.f));
			if(f<=0)continue;
			e.f += f; r.f -= f;
			totf += f;
			a -= f; if(!a)break;
		}
		return totf;
	}

	T operator()(int _s, int _t) {
		s = _s, t = _t;
		T maxf = 0;
		while(bfs()) {
			for(int i=0; i<=n; i++)cur[i] = 0;
			maxf += dfs(s, LINF);
		}
		return maxf;
	}
};
\end{lstlisting}
\subsection{Min Cost Flow}
\begin{lstlisting}
#define ll long long int 
#define LINF 214748364700000LL
#define INF 2147483647
using namespace std;
struct MCF {
	struct edge
	{
		int u, v, c, f;
		ll co;
		edge(int _u, int _v, int _c, ll _co){ u = _u, v = _v, c = _c; co = _co; f = 0; }
	};
	vector<vector<int> > G;
	vector<edge> E;
	vector<ll> d;
	vector<int> inq, arg, p;
	int N, s, t;
	MCF(int _n) {
		N = _n;
		G.resize(_n+1);
		d.resize(_n+1); inq.resize(_n+1);
		arg.resize(_n+1); p.resize(_n+1);
		E.clear();
	}
	void pb(int u, int v, int c, ll co) {
		G[u].push_back(E.size());
		E.push_back(edge(u, v, c, co));
		G[v].push_back(E.size());
		E.push_back(edge(v, u, 0, -co));
	}
	bool BF(int &flow, ll &cost) {
		for(int i=0;i<=N;i++)p[i] = 0, inq[i] = 0, d[i] = LINF;
		queue<int> Q;
		Q.push(s);
		d[s]=0; inq[s] = 1; arg[s] = INF;
		while(!Q.empty()) {
			int x=Q.front(); Q.pop(); inq[x] = 0;
			for(int i=0; i<(int)G[x].size(); i++) {
				edge &e=E[G[x][i]];
				if(d[x] + e.co < d[e.v] && e.c > e.f) {
					d[e.v] = d[x]+e.co;
					p[e.v] = G[x][i];
					arg[e.v] = min(arg[x], e.c - e.f);
					if(!inq[e.v])Q.push(e.v), inq[e.v] = 1;
				}
			}
		}
		if(d[t] == LINF)return 0;
		int a = arg[t];
		for(int now = t; now != s; now = E[p[now]].u) {
			E[p[now]].f += a;
			E[p[now]^1].f -= a;
		}
		cost += arg[t] * d[t];
		flow += a;
		return 1;
	}

	pair<int, ll> operator ()(int _s, int _t) {
		s = _s, t = _t;
		int flow=0;
		ll cost=0;
		while(BF(flow, cost)){}
		return pair<int, ll>(flow, cost);
	}
};
\end{lstlisting}
\subsection{Common Modeling Technique}
\paragraph{Minimum Path Covering on DAG}
\begin{enumerate}
\item Path covering without path intersection: For each vertex $v$, we may construct two vertices $v_i$ and $v_o$, then for each edge $u\rightarrow v$, connect $u_o\rightarrow v_i$. \\
This forms a bipartite graph. Each selected edge means a "join" of paths. Therefore the cardinality of the minimum path covering on the original graph will be $|V| - m$, where $m$ is the cardinality of the maximum bipartite matching.
\item Covering that allows intersection: Perform Floyd-Warshall to obtain trasitive closure first, then make edge for each pair that are connected, the problem subsequently reduces to the non-intersecting case.
\end{enumerate}
\paragraph{Euler Circuit on Undirected Graph}
\begin{enumerate}
\item Give undirected edges directions arbitrary. Add corresponding arc with same direction and capacity $1$ in the network. \\
Filling an edge means "adjust" the direction
\item For each vertex $u$, calculate number of edges need to be changed by direction $d(u) = (deg_{in}(u) - deg_{out}(u))/2$.
\item Add arc from $s$ to each $u$ with $d(u)<0$, to $t$ from each $u$ with $d(u)>0$. The capacity of arc is $|d(u)|$.
\item Check if there exist full flow.
\end{enumerate}
\paragraph{Network with Capacity Lower Bounds} Todo.
\section{Math}
\subsection{ExtGCD}
\begin{lstlisting}
typedef long long int ll;
#define mod 1000000007
void gcd(ll a, ll b, ll &x, ll &y, ll &d) {
	if(!b){ x = 1; y = 0; d = a; return ; }
	gcd(b, a%b, y, x, d); y -= (a/b)*x;
}
ll inv(ll a) {
	ll x, y, d;
	gcd(a, mod, x, y, d);
	return d==1 ? (x+mod)%mod : 0;
}
\end{lstlisting}
\subsection{FFT}
\begin{enumerate}
\itemsep-0.5em
\item When convert back to integer, use \lstinline{LL} can be safer.
\item \lstinline{eps} are \lstinline{0.5} generally, but sometime need adjustments.
\item the array \lstinline{A} and \lstinline{B} will be changed after DFT, and the result \lstinline{AB} has been devided by \lstinline{_n}.
\end{enumerate}
\begin{lstlisting}
#include <stdlib.h>
#include <math.h>
#include <complex>
#include <string.h>
#define MAXN 1048576
#define eps 0.5
#define PI 3.141592653589793238462643383279502884197169399375
#define max(a,b) (((a) > (b)) ? (a) : (b))

typedef std::complex<double> comp;

struct FFT{
	int _n;
	comp ww[MAXN], rw[MAXN];
	void init(int n, int m){ // n terms in polynomial
		_n=1; while(_n<n+m)_n<<=1;
		ww[0] = rw[0] = comp(1.0, 0.0);
		for(int k=1; k<_n; k++){
			ww[k]=comp(cos(2*k*PI/_n), sin(2*k*PI/_n));
			rw[_n-k]=ww[k];
		}
	}
	int rev(int n,int x){int res=0;while(n){res<<=1;res|=x&1;x>>=1;n>>=1;}return res;}
	void dft(int n, comp *res, comp *w){
		for(int i=0; i<n; i++){int j=rev(n>>1,i);if(i<j){comp tmp=res[j];res[j]=res[i];res[i]=tmp;}}
		for(int m=1; m<=n; m<<=1){ 
			if(m==1)continue;
			int mp = m>>1;
			for(int o = 0; o<n; o+=m){
				for(int i=0; i<mp; i++){
					comp tmp = w[i*(n/m)]*res[o+i+mp];
					res[o+i + mp] = res[o+i] - tmp;
					res[o+i] = res[o+i] + tmp;
				}
			}
		}
	}

	void mult(comp *A, comp *B, comp *AB){
		dft(_n, A, ww); dft(_n, B, ww);
		for(int i=0; i<_n; i++)AB[i] = A[i]*B[i];
		dft(_n, AB, rw);
		for(int i=0; i<_n; i++)AB[i]/=_n;
	}
} fft;

comp A[MAXN], B[MAXN];
comp AB[MAXN];
\end{lstlisting}
\subsection{Mobius Function and Sieve}
Given ${p_n}$ be a sequence of distinct primes:
$$
\mu(x)=
\begin{cases}
	1, &\text{if $x=p_1p_2...p_k$, $k$ is even }\\
	-1, &\text{where $k$ above is odd}\\
	0, &\text{if $x$ is not square free}
\end{cases}
$$
\begin{lstlisting}
#define ll long long int
#define MAXN 1000005

ll n;
int isp[1000005];
int mu[1000005];
vector<ll> p;

void sieve() {
	for(int i=0; i<MAXN; i++) isp[i] = 1;
	isp[0] = isp[1] = 0;
	mu[1] = 1;
	for(ll i=2; i<MAXN; i++) {
		if(isp[i]){
			p.push_back(i);
			mu[i] = -1;
		}
		for(int j=0; j<(int)p.size() && i*p[j] < MAXN; j++) {
			ll x = p[j] * i;
			isp[x] = 0;
			if(i % p[j] == 0) {
				mu[x] = 0;
				break;
			}
			mu[x] = -mu[i];
		}
	}
}
\end{lstlisting}
\subsection{Common Theorems}
\paragraph{Josephus Problem}
Let $f(i)$ be the survivor in the round with $i$ people, in the numbering from $0\sim i-1$. Then we have $f(1) = 0$ and $f(i+1) = (f(i)+k) mod (i+1)$. The $+k$ term is to restore the numbering, but stepping through $k$ people. Note that $f(i)$ and $f(i+1)$ used distinct numbering, where $k$th in $f(i+1)$'s is the $0$th in $f(i)$'s.
\paragraph{Pick's Theorem}
For a polygon consist integral-coordinate vertices. Let the number of integral points on the border of the polygon be $a$, and the number of integral points inside the polygon be $b$, then we have the area of the polygon:
$$
A = a+\frac{b}{2}-1
$$
\paragraph{Burnside's Lemma}
$$
|X/G| = \frac{1}{|G|}\sum_{g\in G}{|X_g|}
$$
\paragraph{Mobius Inversion}
For $n\in \mathbb{N}$, if
$$
g(n)=\sum_{d|n}{f(d)}
$$
then
$$
f(n)=\sum_{d|n}{\mu(d)g(n/d)}
$$
\section{Graph}
\subsection{Cut Vertex and BCC}
\paragraph{Determining Bridge}
$low[v] > pre[u] \Rightarrow v$ is a cut vertex and $(u,v)$ is a bridge
\begin{lstlisting}
#define MAXN 1005
using namespace std;
struct edge {
	int u,v;
	edge(int _u,int _v){u=_u;v=_v;}
};
vector<edge> E;
vector<int> G[MAXN];
int N,M;
void pb(int u,int v) {
	G[u].push_back(E.size());
	E.push_back(edge(u,v));
	G[v].push_back(E.size());
	E.push_back(edge(v,u));
}

stack<edge> S;
int pre[MAXN],low[MAXN],bccno[MAXN];
int iscut[MAXN];
int stamp,bcc_cnt;
vector<int> bcc[MAXN];

int dfs(int u,int fa) {
	low[u]=pre[u]=++stamp;
	int ch=0;
	iscut[u]=0;
	for(int i=0;i<(int)G[u].size();i++) {
		edge e=E[G[u][i]];
		int v=e.v;
		if(!pre[v]) {
			ch++;
			S.push(e);
			low[u]=min(low[u],dfs(v,u));
			if(low[v]>=pre[u]) {
				iscut[u]=true;
				bcc_cnt++;
				bcc[bcc_cnt].clear();
				while(1) {
					edge x=S.top();S.pop();
					if(bccno[x.u]!=bcc_cnt)bcc[bcc_cnt].push_back(x.u),bccno[x.u]=bcc_cnt;
					if(bccno[x.v]!=bcc_cnt)bcc[bcc_cnt].push_back(x.v),bccno[x.v]=bcc_cnt;
					if(x.u==u&&x.v==v)break;
				}
			}
		} else if(pre[v]<pre[u]&&v!=fa) {
			S.push(e);
			low[u]=min(low[u],pre[v]);
		}
	}
	if(fa<0&&ch==1)iscut[u]=false;
	return low[u];
}
\end{lstlisting}
\subsection{Kosaraju}
\begin{lstlisting}
#define MAXN 100005
int N;
bool vis[MAXN];
vector<int> G[MAXN];
vector<int> R[MAXN];
vector<int> SCC[MAXN];
int sccno[MAXN];
int scc_cnt;
int owner[MAXN];
int dfs_stamp;

queue<int> Q;
void dfs_for_stamp(int now) {
	vis[now]=true;
	for(int i=0;i<(int)R[now].size();i++)
	{
		int v=R[now][i];
		if(!vis[v]) {
			dfs_for_stamp(v);
		}
	}
	owner[++dfs_stamp]=now;
}

void dfs_for_scc(int now) {
	vis[now]=true;
	sccno[now]=scc_cnt;
	SCC[scc_cnt].push_back(now);
	for(int i=0;i<(int)G[now].size();i++)
	{
		int v=G[now][i];
		if(!vis[v])dfs_for_scc(v);
	}
}

int main() {
	dfs_stamp=0;
	for(int i=1;i<=N;i++) {
		owner[i]=0;
	}
	for(int i=1;i<=N;i++)if(!vis[i])dfs_for_stamp(i);
	for(int i=1;i<=N;i++)vis[i]=false;
	scc_cnt=0;
	for(int i=dfs_stamp;i>=1;i--) {
		if(!vis[owner[i]]){//cout<<i<<" "<<owner[i]<<endl;
			dfs_for_scc(owner[i]),scc_cnt++;
		}
	}
    return 0;
}
\end{lstlisting}
\subsection{Tarjan SCC}
\begin{lstlisting}
vector<int> G[MAXN], scc[MAXN];
vector<int> stk;
int clk, scnt;
int low[MAXN], pre[MAXN], ins[MAXN];

void dfs(int u) {
	ins[u] = 1;
	stk.push_back(u);
	low[u] = pre[u] = ++clk;
	for(int i=0; i<(int)G[u].size(); i++) {
		int v = G[u][i];
		if(!pre[v]) {
			dfs(v);
			low[u] = min(low[u], low[v]);
		} else if(ins[v]) {
			low[u] = min(low[u], pre[v]);
		}
	}
	if(low[u] == pre[u]) {
		scnt++;
		while(stk.size() && stk.back() != u) {
			scc[scnt].push_back(stk.back());
			ins[stk.back()] = 0;
			stk.pop_back();
		}
		if(stk.size() && stk.back() == u) {
			scc[scnt].push_back(u);
			stk.pop_back();
			ins[u] = 0;
		}
	}
}
\end{lstlisting}
\subsection{2-SAT Model}
\paragraph{Problem}
Satisfy the boolean expression like $(x_0 \lor x_1) \land (x_1 \lor \neg x_3)\land ... (x_5 \lor x_2)$
\paragraph{Model}
For the expression $(x_0 \lor x_1)$, make edge $\neg x_0 \rightarrow x_1$, then none of the statements $x_i$ and $\neg x_i$ can be in the same SCC.
\subsection{KM}
\begin{lstlisting}
#define MAXN 1005
#define LL __int128_t
int t,N,K;
LL w[MAXN][MAXN];
LL x[MAXN],y[MAXN];
LL Lx[MAXN],Ly[MAXN];
bool S[MAXN],T[MAXN];
int Left[MAXN];
LL U,L;
const LL INF=(((LL)0x7fffffffffffffLL)<<50)|((LL)0xffffffffffffffffLL);

void getLL(LL &x){
	x=0;
	char c=getchar();
	while(c>'9'||c<'0')c=getchar();
	while(c<='9'&&c>='0'){x*=(LL)10;x+=(LL)(c-'0');c=getchar();}
}
void printLL(LL x){
	if(!x){printf("0"); return ;}
	vector<int> res;
	while(x) {
		res.push_back((int)(x%10));
		x/=(LL)10;
	}
	for(int i=res.size()-1;i>=0;i--)printf("%d",res[i]);
}

void initKM(){
	for(int i=1;i<=N;i++) {
		S[i]=T[i]=false;
		Left[i]=0;
		Lx[i]=Ly[i]=(LL)0LL;
		for(int j=1;j<=N;j++){
			if(w[i][j]==-INF)continue;
			if(x[i]+y[j]>U) {
				w[i][j]=L-U;
			} else if(x[i]+y[j]>L) {
				w[i][j]=L-x[i]-y[j];
			} else w[i][j]=0;
		}
	}
}

bool dfs(int i){
	S[i]=true;
	for(int j=1;j<=N;j++) {
		if(T[j])continue;
		if(Lx[i]+Ly[j]==w[i][j]) {
			T[j]=true;
			if(!Left[j]||dfs(Left[j])) {
				Left[j]=i;
				return true;
			}
		}
	}
	return false;
}

void KM(LL &ANS){
	for(int i=1;i<=N;i++){
		while(true) {
			for(int j=1;j<=N;j++)S[j]=T[j]=0;
			if(dfs(i))break;

			LL d=INF;
			for(int x=1;x<=N;x++){
				if(S[x])
				for(int y=1;y<=N;y++){
					if(!T[y]&&w[x][y]!=-INF)d=min(d,Lx[x]+Ly[y]-w[x][y]);
				}
			}
			if(d==INF){ANS=INF; return ;}

			for(int i=1;i<=N;i++){
				if(S[i])Lx[i]-=d;
				if(T[i])Ly[i]+=d;
			}
		}
	}

	for(int i=1;i<=N;i++) {
		ANS-=w[Left[i]][i];
	}
}

int main(){
	//cout<<INF*2<<endl;
	scanf("%d",&t);
	while(t--){
		scanf("%d",&N);
		getLL(L);getLL(U);
		scanf("%d",&K);
		for(int i=0;i<=N;i++)for(int j=0;j<=N;j++)w[i][j]=0;
		for(int i=0;i<K;i++){
			int u,v;
			scanf("%d%d",&u,&v);
			w[u][v]=-INF;
		}
		for(int i=1;i<=N;i++)getLL(x[i]);
		for(int i=1;i<=N;i++)getLL(y[i]);
		initKM();
		LL ANS=0;
		KM(ANS);
		if(ANS==INF)puts("no");
		else printLL(ANS),puts("");
	}
	return 0;
}
\end{lstlisting}
\subsection{Minimum Mean Cycle}
\paragraph{Remark}
The testcase of the snippet has been modified
\begin{lstlisting}
#define MAXN 5005
#define INF 2147483647000
#define eps 1e-9
#define ll long long int

struct edge {
	int v; ll w;
	edge(int _v, ll _w){v=_v, w=_w;}
};

ll F[MAXN][MAXN];

double MMC(vector<vector<edge> > &G, int n) {
	double ans = 1e9;
	for(int i=0; i<n; i++) F[i][0] = 0;
	for(int i=0; i<n; i++) {
		for(int j=1; j<=n; j++) F[i][j] = INF;
	}
	for(int k=0; k<=n; k++) {
		for(int i=0; i<n; i++) {
			for(int j=0; j<(int)G[i].size(); j++) {
				edge &e = G[i][j];
				F[e.v][k+1] = min(F[e.v][k+1], F[i][k] + e.w);
			}
		}
	}
	for(int i=0; i<n; i++) {
		double tmp = 0;
		for(int k=0; k<n; k++) {
			tmp = max(tmp, (double)(F[i][n] - F[i][k])/(double)(n-k));
		}
		ans = min(ans, tmp);
	}
	return ans;
}
\end{lstlisting}
\section{String}
\subsection{Aho-Corasick Automata}
\begin{lstlisting}
#define MAXN 1000005
template<typename T>
struct AutoAC{
	struct Node {
		int v;
		map<T, Node*> ch;
		typename map<T, Node*>::iterator find(T k){ return ch.find(k); }
		typename map<T, Node*>::iterator begin(){ return ch.begin(); }
		typename map<T, Node*>::iterator end(){ return ch.end(); }
		Node *at(T k){ return ch.at(k); }
		Node *& operator [](T k){ return this->ch[k]; }
		void insert(T k, Node* v){ ch.insert(pair<T, Node*>(k, v)); }

		Node *fail;
	} nodes[MAXN];
	int n;
	Node *root;
	Node *newNode(){ nodes[n].v=0; nodes[n].fail=nullptr; nodes[n].ch.clear(); return nodes+(n++); }
	AutoAC() { n=0; root=newNode(); root->v=0; root->fail=nullptr; }
	void init() { n=0; root=newNode(); root->v=0; root->fail=nullptr; }

	void insert( const T *s , int k ) {
		Node *now = root;
		for(int i=0; s[i]; i++){
			typename map<T, Node*>::iterator it = now->find(s[i]);
			if(it == now->end()){
				now->insert(s[i], newNode());
			}
			now = now->at(s[i]);
		}
		now->v = k;
	}
	void buildFail() {
		queue<Node*> q;
		q.push(root);
		while(!q.empty()) {
			Node *x = q.front(); q.pop();
			for(typename map<T, Node*>::iterator it = x->begin(); it!=x->end(); it++){
				T next = it->first;
				Node *cur = x->fail;
				while(cur&&cur->find(next) == cur->end())cur = cur->fail;
				it->second->fail = cur ? cur->at(next) : root;
				q.push(it->second);
			}
		}
	}
	int search( const T *s ) {
		int res=0;
		Node *cur = root;
		for(int i=0; s[i]; i++){
			while(cur && cur->find(s[i]) == cur->end())cur = cur->fail;
			cur = cur ? cur->at(s[i]) : root;
			if(cur->v)cnt[cur->v]++;
			res = max(cnt[cur->v], res);
		}
		return res;
	}
};
\end{lstlisting}
\subsection{KMP}
\begin{lstlisting}
char s[10005], t[10005];
int f[10005];
// t is 1-based
void buildFail() {
	f[1]=0; f[0]=-1;
	for(int i=2; t[i]; i++){
		int now = f[i-1];
		while(now!=-1 && t[now+1] != t[i])now = f[now];
		f[i] = now+1;
	}
}

int search(char *s, int m) {
	int now = 0, res = 0;
	for(int i=0; s[i]; i++){
		while(now!=-1 && s[i] != t[now+1]) now = f[now];
		now++; 
		if(now == m)res++;
	}
	return res;
}
\end{lstlisting}
\subsection{Suffix Array}
\begin{lstlisting}
#define SIGSZ 130
#define MAXN 1000005
struct SA {
	int c[MAXN];
	int r1[MAXN], r2[MAXN], sa[MAXN], h[MAXN];
	int *rx = r1, *y = r2;
	int neq(int *r, int a, int b, int step, int n){
		return r[a] != r[b] || a+step>=n || b+step>=n || r[a+step] != r[b+step]; 
	}
	void build(int *s, int n, int *_rank, int *_hei, int *_h) {
		for(int i=0; i<SIGSZ; i++) c[i] = 0;
		for(int i=0; i<n; i++)c[rx[i] = s[i]]++;
		for(int i=1; i<SIGSZ; i++)c[i] = c[i-1] + c[i];
		for(int i=n-1; i>=0; i--)sa[--c[s[i]]] = i;
		int m = SIGSZ, p = 0;
		for(int step = 1; step<n; step<<=1, p=0) {
			// storing index of rx[i] based on sorted y[i] to y[i],
			// using the previously calculated sa[i] array.
			for(int i=n-step; i<n; i++)y[p++] = i;
			for(int i=0; i<n; i++)if(sa[i] >= step)y[p++] = sa[i] - step;
			// sorting rx[i] in the order of sorted y[i](aka. rx[y[i]])
			for(int i=0; i<m; i++)c[i] = 0;
			for(int i=0; i<n; i++)c[rx[y[i]]]++;
			for(int i=1; i<m; i++)c[i] = c[i-1] + c[i];
			for(int i=n-1; i>=0; i--)sa[ --c[rx[y[i]]] ] = y[i];
			m = 1; swap(rx, y); rx[sa[0]] = 0;
			for(int i=1; i<n; i++)rx[sa[i]] = neq(y, sa[i], sa[i-1], step, n) ? m++ : m-1;
			if(m == n) break;
		}
		int ph = 0;
		for(int i=0; i<n; i++) h[i] = 0;
		for(int i=0; i<n; i++) {
			if(rx[i] == 0) { h[i] = 0; continue; }
			if(i == 0 || h[i-1] <= 1) {
				for(ph = 0; i+ph<n && s[i+ph] == s[sa[rx[i]-1] + ph]; ph++);
			} else {
				for(ph = h[i-1]-1; i+ph<n && s[i+ph] == s[sa[rx[i]-1] + ph]; ph++);
			}
			h[i] = ph;
		}
		if(_rank){ for(int i=0; i<n; i++)_rank[i] = rx[i]; }
		if(_hei){ for(int i=0; i<n; i++)_hei[i] = h[sa[i]]; }
		if(_h){ for(int i=0; i<n; i++)_h[i] = h[i]; }
	}
	inline int operator [](int i){ return sa[i]; }
};
\end{lstlisting}
\section{Geometry}
\subsection{Convex Hull}
\begin{lstlisting}
// Remember to check if the first point need to be repeated.
#define MAXN 100005
#define ll long long int
struct poi {
	ll x, y;
	bool operator <(const poi &rhs)const {
		return x == rhs.x ? (y < rhs.y) : (x < rhs.x);
	}
};
int test(poi &pi, poi &pj, poi &pk) {
	ll dx1 = pj.x - pi.x, dy1 = pj.y - pi.y;
	ll dx2 = pk.x - pi.x, dy2 = pk.y - pi.y;
	return dx1*dy2 - dx2*dy1 >= 0;
}
void ConvexHull(poi *po, int n, vector<poi> &hull) {
	vector<poi> p;
	for(int i=0; i<n; i++)p.push_back(po[i]);
	sort(p.begin(), p.end());
	hull.push_back(p[0]);
	for(int i=1; i<n; i++) {
		while(hull.size() > 1 && !test(hull[hull.size()-2], hull[hull.size()-1], p[i])) hull.pop_back();
		hull.push_back(p[i]);
	}
	unsigned int h1 = hull.size();
	for(int i=n-2; i>=0; i--) {
		while(hull.size() > h1 && !test(hull[hull.size()-2], hull[hull.size()-1], p[i])) hull.pop_back();
		hull.push_back(p[i]);
	}
	hull.pop_back();
}
\end{lstlisting}
\section{Data Structure}
\subsection{Splay Tree}
\begin{lstlisting}
#define MAXN 200005
#define SZ(o) (o?(o->sz):0)
#define MI(o) (o?(o->minv):2147483647)

struct Node {
	int v,sz;
	int add, minv, rev;
	Node *ch[2];
}NODES[MAXN];
int nodecnt;
Node *newNode() {
	NODES[nodecnt].v=NODES[nodecnt].add=NODES[nodecnt].minv=0;
	NODES[nodecnt].rev = 0;
	NODES[nodecnt].sz=1;
	NODES[nodecnt].ch[0] = NODES[nodecnt].ch[1] = NULL;
	return NODES + (nodecnt++);
}
Node *newNode(int x) { Node *res = newNode(); res->minv = res->v = x; return res; }

void push(Node *&o) {
	if(!o)return ;
	if(o->rev) {
		o->rev = 0;
		swap(o->ch[0], o->ch[1]);
		if(o->ch[0])o->ch[0]->rev ^= 1;
		if(o->ch[1])o->ch[1]->rev ^= 1;
	}
	if(o->add) {
		o->minv += o->add;
		o->v += o->add;
		if(o->ch[0])o->ch[0]->add += o->add;
		if(o->ch[1])o->ch[1]->add += o->add;
		o->add = 0;
	}
}
void pull(Node *&o) {
	if(!o)return ;
	push(o);
	push(o->ch[0]); push(o->ch[1]);
	o->sz = 1;
	o->sz += SZ(o->ch[0]) + SZ(o->ch[1]);
	o->minv = min(o->v, min(MI(o->ch[0]), MI(o->ch[1])));
}
void rotate(Node *&o, int d) {
	push(o);
	Node *c = o->ch[d^1];
	//cout<<o<<" "<<c<<" "<<o->v<<" "<<c->v<<endl;
	push(c);
	o->ch[d^1] = c->ch[d];
	c->ch[d] = o; 
	pull(o); pull(c);
	o = c;
}
void splay(Node *&o, int k) {
	if(!o)return ;
	push(o);
	int i = SZ(o->ch[0]) + 1;
	int d1, d2;
	Node *p;
	//cout<<i<<" "<<k<<endl;
	//cout<<o<<" "<<o->ch[0]<<" "<<o->ch[1]<<endl;
	if(i == k)return ;
	else if(i < k) {
		k -= i;
		d1 = 0;
		p = o->ch[1];
	} else {
		d1 = 1;
		p = o->ch[0];
	}
	push(p);
	i = SZ(p->ch[0]) + 1;
	//cout<<"sec "<<i<<" "<<k<<endl;
	if(i == k){ rotate(o, d1); return ;}
	else if(i < k) {
		k -= i;
		d2 = 0;
		splay(p->ch[1], k);
	} else {
		d2 = 1;
		splay(p->ch[0], k);
	}
	if(d1^d2) { rotate(o->ch[d1^1], d2); rotate(o, d1); }
	else { rotate(o, d1); rotate(o, d2); }
	pull(o);
}
void split(Node *o, int x, Node *&l, Node *&r) {
	if(x == 0) { l = NULL; r = o; return ; }
	push(o);
	splay(o, x);
	r = o->ch[1];
	o->ch[1] = NULL;
	l = o;
	pull(l); pull(r);
}
void merge(Node *&l, Node *r) {
	//cout<<"l->sz: "<<SZ(l)<<endl;
	if(!l){ l = r; return ; }
	splay(l, SZ(l));
	//cout<<"l r "<<SZ(l)<<" "<<SZ(r)<<endl;
	l->ch[1] = r;
	pull(r); pull(l);
}
\end{lstlisting}
\subsection{Link Cut Tree}
\begin{lstlisting}
#define MAXN 50005
struct Node
{
    int tag,sum,color,sz,id;
    Node *p;
    Node *ch[2];
    Node(){tag=sum=color=0;id=0;sz=1;p=ch[0]=ch[1]=NULL;}
    void init(){tag=sum=color=0;id=0;sz=1;p=ch[0]=ch[1]=NULL;}
};
int NODE_ID;
Node NODES[MAXN];
Node* newNode(){NODES[NODE_ID].init();return &NODES[NODE_ID++];}

inline int SZ(Node *o){return o ? o->sz : 0;}
inline int SUM(Node *o){return o ? o->sum : 0;}

void putTag(Node *o, int c){if(!o)return ;o->sum=o->color=o->tag=(1<<c);}

void pull(Node *o){
    if(!o)return ;
    assert(!o->tag);
    o->sz=1+SZ(o->ch[1])+SZ(o->ch[0]); o->sum=o->color|SUM(o->ch[0])|SUM(o->ch[1]);
}

void push(Node *o){
    if(!o)return ;
    if(o->tag){
        o->sum=o->color=o->tag;
        if(o->ch[0])o->ch[0]->color=o->ch[0]->sum=o->ch[0]->tag=o->tag;
        if(o->ch[1])o->ch[1]->color=o->ch[1]->sum=o->ch[1]->tag=o->tag;
        o->tag=0;
    }
}
inline bool isroot(Node *o){return o?(o->p ? (o->p->ch[0]!=o&&o->p->ch[1]!=o) : 1):0;}

void deal(Node *o){
    if(!isroot(o))deal(o->p);
    push(o);
}

Node *A[MAXN];
int ID(Node *o){return o ? o->id : 0;}

void rotate(Node *o, int d){
    Node *t=o->ch[d^1];
    assert(t);
    o->ch[d^1]=t->ch[d]; if(t->ch[d])t->ch[d]->p=o;
    bool notroot=(!isroot(o));
    if(notroot)o->p->ch[o->p->ch[1]==o]=t;
    t->p=o->p;
    t->ch[d]=o; o->p=t;
    pull(o);pull(t);if(notroot)pull(t->p);
}

void splay(Node *o){
    if(!o||isroot(o)){push(o);return ;}
    deal(o);
    int d1,d2;
    while(o->p&&!isroot(o->p)){
        assert(o->p->p);
        d1=(o==o->p->ch[0]);
        d2=(o->p==o->p->p->ch[0]);
        if(d1^d2)rotate(o->p,d1),rotate(o->p,d2);
        else rotate(o->p->p,d2),rotate(o->p,d1);
        if(isroot(o))return ; //bug : o might be root of aux-tree after rotation
    }
    pull(o);
    d1=(o==o->p->ch[0]); //bug : forgot to change the direction
    rotate(o->p,d1);
    assert(isroot(o));
}

void access(Node *o){
    if(!o)return ;

    Node *currentPreferredChain=NULL;
    for(Node *t=o;t;t=t->p){
        splay(t);assert(!t->tag);
        t->ch[1]=currentPreferredChain;
        pull(t);
        currentPreferredChain=t;
    }
    splay(o);
}

Node* find(Node *o){
    access(o);
    for(;o->ch[0];o=o->ch[0]){push(o);}
    return o;
}

Node* cut(Node *o){
    access(o);
    Node *res=o->p ? o->p :NULL;
    if(o->ch[0]){
        for(res=o->ch[0];res->ch[1];res=res->ch[1]);
        o->ch[0]->p=o->p;
        o->ch[0]=o->p=NULL;
    }
    return res;
}

void link_to(Node *x, Node *y, int c){
    if(x==y)return ;
    Node *v=cut(x);
    if(x==find(y)){
        x->p=v; //link back if y is in the subtree of x
    }
    else{
        x->p=y;
        x->color=(1<<c);
    }
}

void query(Node *x, Node *y, int c, int &ans1, int &ans2){
    if(x==y||find(x)!=find(y)){ans1=ans2=0;return ;}
    access(x);
    Node *currentPreferredChain=NULL;

    for(Node *t=y;t;t=t->p){
        splay(t);
        if(!t->p){
            if(c!=-1)putTag(t->ch[1], c),putTag(currentPreferredChain, c);
            else ans1=SZ(t->ch[1])+SZ(currentPreferredChain),ans2=SUM(t->ch[1])|SUM(currentPreferredChain);
        }
        t->ch[1]=currentPreferredChain;
        currentPreferredChain=t;
        pull(t);
    }
    splay(y);
}
\end{lstlisting}
\subsection{Leftist Tree}
\begin{lstlisting}
struct LeftistTree{
	struct Node{
		int v, d;
		Node *l, *r;
		Node(int _v = 0){
			v = _v, d = 1;
			l = r = NULL;
		}
		inline int deep(){
			return d;
		}
		inline void pull(){
			if (!l) {l = r; r=NULL;return ;}
			if (l->deep() < r->deep())
				swap(l, r);
			d = 1 + r->deep();
		}
	}*rt;
	int sz;
	LeftistTree(){
		sz = 0; rt = NULL;
	}
	~LeftistTree(){
		remove(rt);
	}
	Node* merge(Node *L1, Node *L2){
		if (!L1 || !L2) return L1 ? L1 : L2;
		if (L1->v < L2->v){
			L1->r = merge(L1->r, L2);
			L1->pull();
			return L1;
		}else{
			L2->r = merge(L2->r, L1);
			L2->pull();
			return L2;
		}
	}
	void push(int v){
		rt = merge(rt, new Node(v));
		sz++;
	}
	void pop(){
		Node *tmp = rt;
		rt = merge(rt->l, rt->r);
		delete tmp;
		sz--;
	}
	void join(LeftistTree *L){
		rt = merge(rt, L->rt);
		L->rt = NULL;
		sz += L->sz;
		L->sz = 0;
	}
	int size(){ return sz; }
	int top(){ return rt->v; }
	bool empty(){ return !sz; }
	void remove(Node *u){ if (u) remove(u->l), remove(u->r), delete u; }
};
\end{lstlisting}
\subsection{Parallel Binary Search}
\begin{lstlisting}
#define MAXN 100005
#define LL long long int

using namespace std;

int N,M,Q;
LL V[MAXN];
LL S[MAXN];
int A[MAXN];
int ANS[MAXN];

struct query
{
    int l,r,id;LL c;
    bool operator <(const query &r)const{
        return l<r.l;
    }
};

vector<query> QUERIES;
vector<int> FARM[MAXN];
int nxt[MAXN];

LL BIT[MAXN];
void bit_add(int pos,LL v){while(pos<=MAXN)BIT[pos]+=v,pos+=(pos&(-pos));}
LL bit_query(int pos){LL res=0;while(pos>0)res+=BIT[pos],pos-=(pos&(-pos));return res;}

void tot_bs(int s,int e,vector<int>& people)
{
    //cout<<s<<" "<<e<<endl;
    if(s==e){for(auto p:people)ANS[p]=s;return ;}
    vector<query> queries;
    vector<int> farms;
    int mid=(s+e)/2;
    for(int i=0;i<people.size();i++)
    {
        //cout<<"YEE "<<i<<endl;
        int p=people[i];
        for(int j=0;j<FARM[p].size();j++)
        {
            //cout<<"YEE2 "<<j<<endl;
            farms.push_back(FARM[p][j]);
        }
    }
    for(int i=s;i<=mid;i++)
    {
        queries.push_back(QUERIES[i]);
    }
    sort(farms.begin(),farms.end());
    sort(queries.begin(),queries.end());
    map<int,LL> changes;
    int k=0;
    for(auto f:farms)
    {
        for(k;k<queries.size()&&f>=queries[k].l;k++)bit_add(queries[k].r,queries[k].c);
        LL change=bit_query(nxt[f]-1)-bit_query(f-1);
        S[A[f]]+=change;
        if(changes.find(A[f])==changes.end())changes[A[f]]=change;
        else changes[A[f]]+=change;
    }
    vector<int> finished,notyet;
    for(auto p:people)
    {
        if(S[p]>=V[p])finished.push_back(p),S[p]-=changes[p];
        else notyet.push_back(p);
    }
    for(k=k-1;k>=0;k--)bit_add(queries[k].r,-queries[k].c);
    tot_bs(s,mid,finished);
    tot_bs(mid+1,e,notyet);
}

int main()
{
    while(scanf("%d%d%d",&N,&M,&Q)==3)
    {
        vector<int> people;
        for(int i=1;i<=M;i++)
        {
            scanf("%d",A+i);
            FARM[A[i]].push_back(i);
        }
        for(int i=1;i<=N;i++)
        {
            people.push_back(i);
            for(int j=0;j<FARM[i].size();j++)
            {
                nxt[FARM[i][j]]=(j+1==FARM[i].size() ? M+1 : FARM[i][j+1]);
            }
            scanf("%lld",V+i);
        }
        for(int i=0;i<Q;i++)
        {
            int l,r;LL c;
            scanf("%d%d%lld",&l,&r,&c);
            QUERIES.push_back((query){l,r,i,c});
        }
        tot_bs(0,Q,people);
        for(int i=1;i<=N;i++)
        {
            printf("%d\n",ANS[i]==Q?-1:ANS[i]+1);
        }
    }
    return 0;
}
\end{lstlisting}
